\section{Feature Engineering}
Our team members extracted 7 feature sets, namely F1, F2, F3, F4, F5, F6, and F7 from raw data independently.

\subsection{Data Sets}
Activity logs of 200,906 enrollments from 112,448 students across 39 courses are provided.
Each activity is described by 6 fields of the username, course ID, timestamp, source, event, and object. 
For each object, 3 additional fields of the category, children, and start date are provided.
The training set consists of 8,157,278 logs from 120,543 enrollments with the target variable indicating if a student dropped out.  
The test set consists of 5,387,848 logs from 80,363 enrollments.
The full description of the data sets is available in \cite{kddcup2015_data}. In general, this data can be organized in three dimension space, object, time, and event as shown in Figure~\ref{fig:cube}. Feature engineering tasks were carried out based on these views.

\begin{figure}[!t]
	\caption{Data Cube}
	\centering
	\includegraphics[width=0.5 \textwidth]{cube}
	\label{fig:cube}
\end{figure}


\subsection{Common Features}
There are common features across 7 feature sets as follows:
\begin{itemize}
	\item Number of objects
	\item Number of events
	\item Aggregation features these count features
\end{itemize}
The common features can be generated using cube operations. Figure~\ref{fig:slice} shows an example how weekly and monthly count features are calculated. Firstly, the data is cut using object dimension. In this case, we choose to generate feature for users. Next, we select an event "navigate" in the event space to generate a time series presenting "navigate" event over the time. Finally, drill down operation is used to generate monthly or weekly count features.

\begin{figure}[!t]
	\caption{Slice and Dice}
	\centering
	\includegraphics[width=0.5 \textwidth]{slice_and_dice}
	\label{fig:slice}
\end{figure}

\subsection{F1}

Features generated by Song and Kohei can be classified as follows:

\begin{itemize}
  \setlength\itemsep{0em}
  \item Enrollment-based features (No.1-8)
  \item Username-based features (No.9-18)
  \item Username-based features for each courses (No.19-25) 
  \item Features based on 10 days after the end date of course (No.26-35)
  \item Features based on 1 day after the end date of a course (No.36-45)
  \item Day-level features (No.46)
  \item Day-level features using target variables (No.47-58)
\end{itemize}

Full list of features generated by Song and Kohei are described in Table~\ref{tb:skfeature}.

\subsection{F2}
Peng and Xiaocong features are comprised of the following parts:
\begin{itemize}
  \setlength\itemsep{0em}
  \item Visit time(hour, day) set features (including time span and max absent days)
  \item Act(event, object) counting features (some uses missed content counts)
  \item Course drop rate
  \item Number of courses the user enrolled
  \item Minimum time interval between time points(first visit, last visit, course begin, course end, 10 days after course end) of current course and another enrolled course
  \item Active days between course end and 10 days after course end
  \item Active days between last visit and course end
  \item Number of courses ended after current course end
\end{itemize}

The full feature list could be found in Table~\ref{tb:rwfeature}.

\subsection{F3}
These features were generated by Tam. They can be categorized into three major groups, count, aggregation, and date features. The list of features is as follows:
\subsubsection{Count Feature}
There are a few entities such as user, course, and object in the training dataset. Combining these entities together, we have user activities or events. The simplest way to generate features from these events is to count the number of times an entity engaging in the event. The motivation is that the more does a user participate in course, the more chance does he drop out that course. The list of count features are given in Table~\ref{tb:tnfeature1}.

\begin{center}
	\begin{table*}[ht]
		\begin{minipage}{\textwidth}
			{
				\small
				\hfill{}
				\begin{tabular}{|l|l|l|}
					\hline
					\textbf{No.}&\textbf{Feature}&\textbf{Description}\tabularnewline \hline
					1 & User counts & The log count of each user \tabularnewline
					2 & Course count & The log count of each course \tabularnewline
					3 & Event count & The log count of each event \tabularnewline
					4 & User weekly count & The log count of each user per week \tabularnewline
					5 & User bi-weekly count & The log count of each user per two weeks \tabularnewline
					6 & User weekday count & The log count of each user per weekday \tabularnewline
					7 & User monthly count & The log count of each user per month\tabularnewline
					8 & Course weekly count & The log count of each course per week \tabularnewline
					9 & Course bi-weekly count & The log count of each course per two weeks \tabularnewline
					10 & Course weekday count & The log count of each course per weekday\tabularnewline
					11 & Course monthly count & The log count of each course per month \tabularnewline
					12 & Event weekly count & The log count of each event per week \tabularnewline
					13 & Event bi-weekly count & The log count of each event per two weeks \tabularnewline
					14 & Event weekday count & The log count of each event per weekday \tabularnewline
					15 & Event monthly count & The log count of each event per month \tabularnewline
					\hline
				\end{tabular}
			}
			\hfill{}
			\caption{List of count features generated by Tam.}
			\label{tb:tnfeature1}
		\end{minipage}
	\end{table*}
\end{center}

\subsubsection{Aggregation Feature}
Aggregation features were calculated based on count features. Usually, each course would have a fixed schedule for users to study. Therefore, students roll in the course must have stable activity patterns. Aggregation features would measure the stability of course engagement. These features are mean, median, standard deviation of count on date basis such as weekly, monthly, etc. The list of aggregation features are given in Table~\ref{tb:tnfeature2}.

\begin{center}
	\begin{table*}[ht]
		\begin{minipage}{\textwidth}
			{
				\small
				\hfill{}
				\begin{tabular}{|l|l|l|}
					\hline
					\textbf{No.}&\textbf{Feature}&\textbf{Description}\tabularnewline \hline
					1 & Min & Min of all above count features \tabularnewline
					2 & Max & Max of all above count features \tabularnewline
					3 & Mean & Mean of all above count features \tabularnewline
					4 & Median & Median of all above count features \tabularnewline
					5 & Std & Standard deviation of all above count features  \tabularnewline
					\hline
				\end{tabular}
			}
			\hfill{}
			\caption{List of aggregation features generated by Tam.}
			\label{tb:tnfeature2}
		\end{minipage}
	\end{table*}
\end{center}

\subsubsection{Date Feature}
To capture how often users participate in a certain course, we generated date features. Date features can be time span among user activities as well as time span from last activity and last course date. The list of date features is given in Table~\ref{tb:tnfeature3}.

\begin{center}
	\begin{table*}[ht]
		\begin{minipage}{\textwidth}
			{
				\small
				\hfill{}
				\begin{tabular}{|l|l|l|}
					\hline
					\textbf{No.}&\textbf{Feature}&\textbf{Description}\tabularnewline \hline
					1 & Min time span & Min time span among activities \tabularnewline
					2 & Max time span & Max time span among activities \tabularnewline
					3 & Mean time span & Mean time span among activities \tabularnewline
					4 & Last time span & Time span from the last activity and last course date \tabularnewline
					5 & Number of unique days & The number of unique activity days of each user \tabularnewline
					\hline
				\end{tabular}
			}
			\hfill{}
			\caption{List of date features generated by Tam.}
			\label{tb:tnfeature3}
		\end{minipage}
	\end{table*}
\end{center}

\subsection{F4}
Features generated by Michael Jahrer are in sparse format:

\begin{itemize}
  \setlength\itemsep{0em}
  \item uID (0-112,447)
  \item cID (112,448-112,486)
  \item uIDcnt (112,487-112,487)
  \item eIDcnt (112,488-112,488)
  \item eID $\rightarrow$ sID (112,489-112,490)
  \item eID $\rightarrow$ evID (11,2491-112,497)
  \item eID $\rightarrow$ oIDCnt (112,498-139,443)
  \item eID $\rightarrow$ tIDCnt (139,444-139,635)
  \item uID: floor(log(dateSpan$^2$+1)) (139,636-140,635)
  \item uID $\rightarrow$ log(time diff to obj start+1) (140,636-140,636)
  \item eID $\rightarrow$ dateVec diff stats (140,637-140,649)
\end{itemize}

\subsection{F5}
\begin{itemize}
  \setlength\itemsep{0em}
  \item Course ID - One-hot-encoded course\_id
  \item Source time counts  by enrollment - The log count of each source type per day for each enrollment
  \item Source time counts by course id - The log count of each source type per day for each course id
  \item Event time counts by enrollment - The log count of each event type per day for each enrollment
  \item Event time counts by course id - The log count of each event type per day for each course id

\end{itemize}

\subsection{F6}
Features generated by Jeong-Yoon Lee are as follows:

\begin{itemize}
  \setlength\itemsep{0em}
  \item User ID (20,113) - One-hot-encoded username. Usernames appearing less than 100 times in training log data are grouped together as one user ID. 
  \item Course ID (39) - One-hot-encoded course\_id.
  \item Source Event (10) - One-hot-encoded combination of source and event.
  \item Object ID (3,554) - One-hot-encoded object.  Objects appearing less than 100 times in training log data are grouped together as one object ID.
  \item Count (1) - Number of log entries for an hour\_id.
  \item Object Category (6) - Number of log entries with an object category for an enrollment\_id.
  \item Number of Children Objects (7) - One-hot-encoded total number of object's children for an enrollment\_id.
  \item Object Timespan (10) - One-hot-encoded timespan in days between object's start date and last day of the class
  \item Day of Class (30) - One-hot-encoded day of the class
  \item Week of Class (4) - One-hot-encoded week of the class
  \item End Month of Class (7) - One-hot-encoded end month of the class
  \item Object Started in Dropout Period (2) - Binary variable that is 1 if object started after but before 10 days after last day of the class and 0 otherwise.
\end{itemize}

\subsection{F7}

F1 features and additional features generated by Kohei. Additional features are focused on encoding target variables for each days.

\begin{itemize}
  \setlength\itemsep{0em}
  \item For each 10 days after the end date of the course, number of active enrollment\_id, which target variables are 1 in the training set, enrolled by an username.
  \item For each 10 days after the end date of the course, number of active enrollment\_id, which target variables are 0 in the training set, enrolled by an username.
  \item For each 10 days after the end date of the course, number of active enrollment\_id (in this case, days between last access and the end date of the course are also counted for active days), which target variables are 1 in the training set, enrolled by an username.
  \item For each 10 days after the end date of the course, number of active enrollment\_id (in this case, days between last access and the end date of the course are also counted for active days), which target variables are 0 in the training set, enrolled by an username.
  \item For each 14 days before the end date of the coruses, number of active enrollment\_id, which target variables are 1 in the training set, enrolled by an username.
  \item For each 14 days before the end date of the coruses, number of active enrollment\_id, which target variables are 0 in the training set, enrolled by an username.
  \item For each 14 days before the end date of the coruses, number of active enrollment\_id (in this case, days between last access and the end date of the course are also counted for active days), which target variables are 1 in the training set, enrolled by an username.
  \item For each 14 days before the end date of the coruses, number of active enrollment\_id (in this case, days between last access and the end date of the course are also counted for active days), which target variables are 0 in the training set, enrolled by an username.
\end{itemize}
